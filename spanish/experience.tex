\section{Experiencia Laboral}
  \subsection{Vocacional}
    \cventry{2016--Presente}{Desarrollador Web y Sistemas}{Aseguradora Porvenir}{http://www.aspor.cl}{Santiago}{Desarrollo en equipo del sistema para emitir pólizas y endozos, permitiendo su consulta y el posterior pago de las cuotas generando el arqueo de caja. Sistema desarrollado utilizando Java Server Faces y Primefaces, en conjunto con Bootstrap y CSS personalizado para el diseño.\newline{}%
      Encargado del desarrollo y mantención de la página web utilizando Grunt para autómatizar el compilado de Sass, minificación de javascripts y optimización de imagenes.
    }
    \cventry{2014--2016}{Desarrollador Web}{AgendaPro}{http://www.agendapro.co}{Santiago}{Sistema de gestión y reservas en línea para pequeñas y medianas empresas que presten servicios.\newline{}%
      Desarrollo full-stack en Ruby on Rails utilizando jQuery y Bootstrap. Principales módulos desarrollados:%
      \begin{itemize}%
        \item \textit{Workflow} para poder tomar horas en línea, incluyendo vista de horas disponibles.
        \item Múltiples servicios para una misma reserva y optimizador de horas para encontrar el mejor momento para realizarlos.
        \item Sistema de emails para notificar nuevas reservas y recordatorios.
        \item Gestión de Clientes:
        \begin{itemize}%
          \item Vista e historial de reservas.
          \item Filtros para buscar clientes.
          \item Sistema de emails promocionales.
        \end{itemize}
        \item Sistema de búsqueda de locales, servicios o prestadores.
      \end{itemize}
    }
    \cventry{2012--2014}{Desarrollador Web}{ICCSolutions}{Santiago}{}{Fabrica de Softwares de la Universidad de los Andes\newline{}%
      Desarrollo de distintos softwares a pedidos basados en Web2py:
      \begin{description}
        \item[FAI] Software para el control y gestión de recursos para el departamento de Investigación y Desarrollo.
        \item[IngUandes] Página de cursos para la Facultad de Ingeniería y Ciencias aplicadas. Módulos desarrollados:
        \begin{itemize}
          \item Postulación, entrega de informe y evaluación de prácticas.
          \item Cartas al consejo.
          \item Postulación a ayudantías.
        \end{itemize}
        \item[SAF] Adaptación de \textit{IngUandes} a otras facultades y desarrollo de nuevos módulos.
      \end{description}
    }
  \subsection{Misceláneos}
    \cventry{2011--2013}{Ayudante Corrector}{Facultad de Ingeniería y Ciencias aplicadas}{Santiago}{}{Corrector de pruebas y trabajos del ramo Introducción a la Programación.}
    \cventry{2013 Segundo Semestre}{Ayudante de Cátedra}{Facultad de Ingeniería y Ciencias aplicadas}{Santiago}{}{Ayudante de cátedra del ramo Introducción a la Programación.}
    \cventry{2014 Primer Semestre}{Ayudante Corrector}{Facultad de Ingeniería y Ciencias aplicadas}{Santiago}{}{Corrector de pruebas y trabajos del ramo Tecnologías de la Web.}
    \cventry{2009--2014}{Patrulla UAndes}{Departamento de Admisión y Promoción}{Universidad de los Andes}{Santiago}{Apoyo en la promoción de la Universidad de los Andes en diversas actividades realizadas para los futuros alumnos.}
